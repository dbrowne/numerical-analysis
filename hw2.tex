%Format
\documentclass[12pt]{article}

%Packages
\usepackage{amsfonts}
\usepackage{amsmath}

%Margins
\setlength{\topmargin}{-.5in}
\setlength{\textheight}{9in}
\setlength{\oddsidemargin}{.125in}
\setlength{\textwidth}{6.25in}

% Shortcuts
\newcommand{\eqn}[2]{\begin{equation}#1\label{#2}\end{equation}}
\newcommand{\sol}[2]{\\ \fbox{\parbox{\linewidth}{\textbf{Solution:} #1}}\\}

% Variables
\newcommand{\student}{David Palma}
\newcommand{\class}{MATH 414: Numerical Analysis}
\newcommand{\assignment}{Homework 2}
\newcommand{\duedate}{2011 Sep 26}
\newcommand{\instructor}{Dr. D. Meng}

\title{\class \\ \instructor}
\author{\student \\ Graduate \\ \assignment}
\date{\duedate}

% Document
\begin{document}

%Titlepage
\maketitle
\newpage

% Problem Set
\begin{enumerate}
    
    \item 1.3 \#3 The Maclaurin series for the arctangent function converges for ${-1<x\leq1}$ and is given by
    \\ ${\arctan{x}=\displaystyle\lim_{n\to\infty}P_n(x)=\displaystyle\lim_{n\to\infty}\displaystyle\sum_{i=1}^n (-1)^{i+1}\frac{x^{2i-1}}{2i-1}}$.
    
    (a) Use the fact that ${\tan{\pi/4=1}}$ to determine the number of ${n}$ terms of the series that need to be summed to ensure that ${|4P_n(1)-\pi|<10^{-3}}$. \sol{}

    (b) The C++ programming language requires the value of ${\pi}$ to be within ${10^{-10}}$. How many terms of the series would you need to sum to obtain this degree of accuracy. \sol{}

    \item 1.3 \#6 Find the rates of convergence of the following sequences as ${n\to\infty}$.

    (a) ${\displaystyle\lim_{n\to\infty}\sin{1/n}=0}$. \sol{}
    
    (c) ${\displaystyle\lim_{n\to\infty}\sin(1/n)^2=0}$. \sol{}

    \item 1.3 \#7 Find the rates of convergence of the following functions as ${h\to\infty}$.

    (b) ${\displaystyle\lim_{h\to0} \frac{1-\cos{h}{h}}=0}$. \sol{}

    (d) ${\displaystyle\lim_{h\to0} \frac{1-e^h}{h}=-1}$. \sol{}

    \item 1.3 \#15 Suppose that as ${x}$ approaches zero, 
    \\ ${F_1(x)}=L_1+O(x^\alpha)$ and ${F_2(x)=L_2+O(x^\beta)}$.
    \\ Let ${c_1}$ and ${c_2}$ be nonzero constants, and define 
    \\ center{}
    \\ center {}.
    \\ Show that if ${\gamma=}$minimum ${\{a,b\}}$, then as ${x}$ approaches zero,

    (a) ${F(x)=c_1L_1+c_2L_2+O(x^\gamma)}$. \sol{}

    (b) ${G(x)=L_1+L_2+O(x^\gamma)}$ \sol{}

    \item 1.3 \#
\end{enumerate}

\end{document}

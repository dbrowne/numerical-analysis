%Format
\documentclass[12pt]{article}

%Packages
\usepackage{amsfonts}
\usepackage{amsmath}

%Margins
\setlength{\topmargin}{-.5in}
\setlength{\textheight}{9in}
\setlength{\oddsidemargin}{.125in}
\setlength{\textwidth}{6.25in}

% shortcuts
\newcommand{\eqn}[2]{\begin{equation}#1\label{#2}\end{equation}}

\begin{document}
%\maketitle
\center{MATH 414: Homework 1}
\center{David Palma}

\begin{enumerate}
    \item 1.1 \#1 Show that the following equations have at least one solution in the given intervals.

    (d) ${x-(\ln{x})^x=0, [4,5]}$ \\ \textbf{Solution:}
    Let ${f(x)=x-(\ln{x})^x}$. The function ${f}$ is continuous on [4,5]. Furthermore, \\
        ${f(4)=4-(ln{4})^4 \approx 4-3.69>0}$ \\
        ${f(5)=5-(ln{5})^5 \approx 5-10.8<0}$ \\

    By the Intermediate Value Theorem, there exists a real number \(c\),${4<c<5}$, such that ${f(c)=0}$.

    \item 1.1 \#2 Find the intervals containing solutions to the following equations.

    (c) ${x^3-2x^2-4x+2=0}$ \\ \textbf{Solution:}
    Let ${f(x)=x^3-2x^2-4x+2}$. ${f(x) \in C(-\infty,\infty)}$

    \item 1.1 \#4 Find max${_{a \leq x \leq b}|f(x)|}$ for the following functions and intervals.

    (a) ${f(x)=(2-e^x+2x)/3, [0,1]}$ \\ \textbf{Solution:}

    \item 1.1 \#9 Find the second Taylor polynomial ${P_2(x)}$ for the function ${f(x)=e^x\cos{x}}$ about ${x_0=0.}$

    (a) Use

    (b)

    (c)

    (d)

    \item 1.1 \#14 Use the error term of a Taylor polynomial to estimate the error involved in using ${\sin{x} \approx x}$ to approximate ${\sin{1^\circ}}$

    \item 1.1 \#28 Suppose ${f \in C[a,b]}$, that ${x_1}$ and ${x_2}$ are in [a,b].

    (a)

    (b)

    (c)

    \item 1.2 \#5 Use three-digit rounding arithmetic to perform the following calculations. Compute the absolute error and relative error with the exact value determined to at least five digits.

    (e)

    \item 1.2 \#9 The first three nonzero terms of the Maclaurin series for the arctangent function are ${x-(1/3)x^3+(1/5)x^5}$. Compute the absolute error and relative error in the following approximations of ${\pi}$ using the polynomial in place of the arctangent:

    (a)

    \item 1.2 \#13 Use four-digit rounding arithmetic and the formulas in (1.1), (1.2), and (1.3) to find the most accurate approximations to the roots of the following quadratic equations. Compute the absolute errors and relative errors.

    (c)

    \item 1.2 \#26 Let ${f \in C[a,b]}$ be a function whose derivative exists on (a,b). Suppose ${f}$ is to be evaluated at ${x_0}$ in (a,b), but instead of computing the actual value ${f(x_0)}$, the approximate value, ${\tilde{f}(x_0)}$, is the actual value of ${f}$ at ${x_0+\epsilon}$, that is, ${\tilde{f}(x_0)=f(x_0+\epsilon)}$.

\end{enumerate}

\end{document}
